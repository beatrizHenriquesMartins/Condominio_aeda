\documentclass[a4paper]{article}

%use the english line for english reports
%usepackage[english]{babel}
\usepackage[portuguese]{babel}
\usepackage[utf8]{inputenc}
\usepackage{indentfirst}
\usepackage{graphicx}
\usepackage{verbatim}


\begin{document}

\setlength{\textwidth}{16cm}
\setlength{\textheight}{22cm}

\title{\Huge\textbf{Gestão de um Condomínio}\linebreak\linebreak\linebreak
\Large\textbf{Relatório}\linebreak\linebreak
\linebreak\linebreak
\includegraphics[scale=0.1]{feup-logo.png}\linebreak\linebreak
\linebreak\linebreak
\Large{Mestrado Integrado em Engenharia Informática e Computação} \linebreak\linebreak
\Large{Algoritmos e Estruturas de Dados}\linebreak
}

\author{\textbf{Grupo:}\\
Adelaide Silva - 201406986@fe.up.pt \\
Beatriz Martins - up201502858@fe.up.pt \\
Maria Teresa Chaves - up201306842@fe.up.pt
\linebreak\linebreak \\
 \\ Faculdade de Engenharia da Universidade do Porto \\ Rua Roberto Frias, s\/n, 4200-465 Porto, Portugal \linebreak\linebreak\linebreak
\linebreak\linebreak\vspace{1cm}}

\maketitle
\thispagestyle{empty}

% ------------------------------------------------------------------
% ------------------------------------------------------------------

\newpage

% ------------------------------------------------------------------
\tableofcontents

\newpage

%Todas as figuras devem ser referidas no texto. %\ref{fig:codigoFigura}
%
%%Exemplo de código para inserção de figuras
%%\begin{figure}[h!]
%%\begin{center}
%%escolher entre uma das seguintes três linhas:
%%\includegraphics[height=20cm,width=15cm]{path relativo da imagem}
%%\includegraphics[scale=0.5]{path relativo da imagem}
%%\includegraphics{path relativo da imagem}
%%\caption{legenda da figura}
%%\label{fig:codigoFigura}
%%\end{center}
%%\end{figure}
%
%
%\textit{Para escrever em itálico}
%\textbf{Para escrever em negrito}
%Para escrever em letra normal
%``Para escrever texto entre aspas''
%
%Para fazer parágrafo, deixar uma linha em branco.
%
%Como fazer bullet points:
%\begin{itemize}
	%\item Item1
	%\item Item2
%\end{itemize}
%
%Como enumerar itens:
%\begin{enumerate}
	%\item Item 1
	%\item Item 2
%\end{enumerate}
%
%\begin{quote}``Isto é uma citação''\end{quote}


% ------------------------------------------------------------------
\section{Introdução}

\newpage

% ------------------------------------------------------------------
\section{Gestão de um Condomínio}

\newpage

% ------------------------------------------------------------------
\section{Solução implementada}

\newpage

% ------------------------------------------------------------------
\section{Diagrama UML}

\newpage

% ------------------------------------------------------------------
\section{Casos de utilização}

\newpage

% ------------------------------------------------------------------
\section{Principais dificuldades encontradas}

\newpage

% ------------------------------------------------------------------
\section{Distribuição do Esforço Dedicado}

\newpage

\end{document}
