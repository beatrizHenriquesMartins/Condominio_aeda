\documentclass[a4paper]{article}

\usepackage[portuguese]{babel}
\usepackage[utf8]{inputenc}
\usepackage{indentfirst}
\usepackage{graphicx}
\usepackage{verbatim}
\usepackage{listings}
\usepackage{xcolor}

\begin{document}

\setlength{\textwidth}{16cm}
\setlength{\textheight}{22cm}

\title{\Huge\textbf{Gestão de um Condomínio}\linebreak\linebreak\linebreak
\Large\textbf{Relatório Final}\linebreak\linebreak
\linebreak\linebreak
\includegraphics[scale=0.1]{feup-logo.png}\linebreak\linebreak
\linebreak\linebreak
\Large{Mestrado Integrado em Engenharia Informática e Computação} \linebreak\linebreak
\Large{Algoritmos e Estruturas de Dados}\linebreak
}

\author{\textbf{Grupo:}\\
Adelaide Silva - 201406986@fe.up.pt \\
Beatriz Martins - up201502858@fe.up.pt \\
Maria Teresa Chaves - up201306842@fe.up.pt
\linebreak\linebreak \\
 \\ Faculdade de Engenharia da Universidade do Porto \\ Rua Roberto Frias, s\/n, 4200-465 Porto, Portugal \linebreak\linebreak\linebreak
\linebreak\linebreak\vspace{1cm}}

\maketitle
\thispagestyle{empty}

% ------------------------------------------------------------------
% ------------------------------------------------------------------

\newpage

% ------------------------------------------------------------------
\tableofcontents

\newpage

% ------------------------------------------------------------------
\section{Descrição do Tema}

O projeto realizado sobre o tema "Gestão de um Condomínio" consistiu em desenvolver uma aplicação em C++ que pudesse ser hipoteticamente utilizada por uma empresa de condomínios, de forma a gerir as habitações de cada condomínio, proceder ao pagamento das mensalidades dos seus residentes (clientes), gerir os serviços necessários que podem ser prestados a cada habitação (neste caso, apenas é possível Limpeza, Canalização e Pintura), e contratar e remover empregados.

A interface desenvolvida permite aos utilizadores escolherem o condomínio que pretendem gerir, alterarem e pesquisar dados sobre habitações, clientes, empregados e serviços prestados a um dado condomínio.

\subsection{Objetivos Gerais}

Os objetivos gerais da aplicação são os seguintes:

\begin{itemize}

	\item TODO: Ver primeira parte do enunciado
	\item Guardar os condomínios que a empresa de condomínios administra numa árvore binária de pesquisa, sendo a ordenação efetuada por número de habitações e, em caso de empate, por número de vivendas;
	\item Dispor de sistema de transportes públicos, sendo que estes devem ser armazenados numa fila de prioridade;
	\item Guardar os prestadores de serviços do condomínio numa tabela de dispersão, devendo ser permitido listagens ou pesquisas sobre estes dados.

\end{itemize}

\subsection{Descrição da Aplicação}

TODO: Indicar algumas secções que "liguem conceitos", por exemplo, Serviços e Empregados, Clientes e Habitações, etc.

\subsubsection{Clientes e Habitações}

\subsubsection{Empregados e Serviços}

\newpage

% ------------------------------------------------------------------
\section{Solução Implementada}

\subsection{Análise}

Após a análise do enunciado, devido a algumas dúvidas perante certas ambiguidades, foram assumidas as seguintes condições:

\begin{itemize}

	\item Um \textbf{Servico} (classe) não é um serviço, mas sim uma empresa de serviços, composta por vários empregados armazenados numa tabela de dispersão;
	\item TODO: acrescentar mais aspetos que foram assumidos...

\end{itemize}

\subsection{Classes}

Tendo em consideração os requisitos da aplicação, foram desenvolvidas as seguintes classes de forma a cumprir os objetivos pré-estabelecidos.

\subsubsection{Apartamento}

Esta classe tem como objetivo guardar informações sobre um apartamento, herdadas da classe Habitação, tais como a morada da habitação e a sua área habitacional. Além disto, é possível armazenar dados sobre a tipologia e piso.

\subsubsection{Canalizacao}

Esta classe tem como objetivo guardar informações essencias sobre um serviço de \textbf{Pintura}, herdadas da classe \textbf{Empregado}, tais como o nome do empregado, o seu número de identificação, o número de telemóvel, etc.

\subsubsection{Condominio}

Esta classe tem como objetivo guardar informações essenciais sobre o próprio condomínio, tais como o nome, o número de identificação fiscal, contatos telefónicos e \textit{email}, os clientes existentes e informação sobre a empresa de serviços. Além disto é possível requisitar ou terminar um serviço de Limpeza, Canalização ou Pintura, ou adicionar e remover um dado cliente.

\subsubsection{Cliente}

Esta classe tem como objetivo guardar informações essenciais sobre um cliente, tais como o nome, o seu número de identificação, contatos telefónicos e \textit{email}, e respetivas habitações. Um dado cliente pode adicionar uma nova habitação (introduzindo os respetivos dados, i.e, a morada, a área de habitação, etc.) ou remover uma que já possua, sendo apenas necessário indicar a morada da habitação que se pretende remover.

\subsubsection{Empregado}

Esta classe tem como objetivo guardar informações essenciais sobre um empregado, tais como o nome, \textit{email}, número de telemóvel, o seu tipo (\textbf{Limpeza}, \textbf{Canalização} ou \textbf{Pintura}) e se está disponível de momento. No caso de um dado empregado não estar disponível, então o condomínio não poderá requisitá-lo para um serviço até que este termine. A aplicação irá tentar satisfazer um pedido do condomínio (e.g serviço de limpeza), procurando se existe pelo menos um empregado disponível desse tipo.

\subsubsection{Interface}

Esta classe tem como objetivo disponibilizar ao utilizador os menús essenciais da aplicação, ler os dados necessários dos ficheiros dos vários condomínios (ficheiro do próprio condomínio, dos clientes, empregados, habitações e serviços), proceder a pesquisas de informação básica (por nome, por tipo, etc.), edição de informação (e.g. tipologia no caso de ser um apartamento) e gestão do fluxo de dados e armazenamento do estado atual da aplicação, aquando da sua terminação.

\subsubsection{Habitacao}

Esta classe tem como objetivo guardar informações essenciais sobre uma habitação, tais como a sua morada, a sua área e os serviços que foram prestados a. Para além de ser possível armazenar esta informação, também é possível adicionar um serviço (\textbf{Limpeza}, \textbf{Canalização} ou \textbf{Pintura}), sendo que a aplicação tentará satisfazer essa necessidade e, por último, obter informações básicas. Uma habitação pode ser considerada uma \textbf{Vivenda} ou um \textbf{Apartamento}.

\subsubsection{Limpeza}

Esta classe tem como objetivo guardar informações essencias sobre um serviço de \textbf{Limpeza}, herdadas da classe \textbf{Empregado}, tais como o nome do empregado, o seu número de identificação, o número de telemóvel, etc.

\subsubsection{Pintura}

Esta classe tem como objetivo guardar informações essencias sobre um serviço de \textbf{Pintura}, herdadas da classe \textbf{Empregado}, tais como o nome do empregado, o seu número de identificação, o número de telemóvel, etc.

\subsubsection{Servico}

Esta classe tem como objetivo guardar informações sobre o número de serviços disponíveis, o número máximo de empregados de \textbf{Limpeza}, \textbf{Canalização} e \textbf{Pintura} e os empregados efetivos da empresa de serviços que são armazenados numa tabela de dispersão. Esta classe, além de permitir a inserção e remoção de empregados, garante a integridade dos dados, a não existência de empregados duplicados na tabela de dispersão e serve como base de informação para a classe \textbf{Condominio}. 

\subsubsection{Vivenda}

Esta classe tem como objetivo guardar informações sobre uma vivenda, herdadas da classe Habitação, tais como a morada da habitação e a sua área habitacional. Além disto, é possível armazenar dados sobre a área exterior e a possibilidade de ter uma piscina ou não.

\newpage

% ------------------------------------------------------------------
\section{Casos de Utilização}

% Lista de Casos de Utilização identificadas para a aplicação (não é necessário desenhar os diagramas de casos de utilização)

\newpage

% ------------------------------------------------------------------
\section{Diagrama UML}

\newpage

% ------------------------------------------------------------------
\section{Principais Dificuldades Encontradas}

% Relato das principais dificuldades encontradas no desenvolvimento do trabalho

\newpage

% ------------------------------------------------------------------
\section{Distribuição do Esforço Dedicado}

O esforço dedicado por cada elemento do grupo é o seguinte: \newline

Adelaide Nelma da Costa Ferreira da Silva - \newline

Beatriz de Henriques Martins - \newline

Maria Teresa dos Santos Carneiro Chaves - 

\newpage

\end{document}
